\documentclass[12pt,a4paper]{article}
\usepackage[UTF8]{ctex}
\usepackage{amsmath}
\usepackage{amssymb}
\usepackage{graphicx}
\usepackage{booktabs}
\usepackage{multirow}
\usepackage{listings}
\usepackage{xcolor}
\usepackage{geometry}
\usepackage{float}
\usepackage{subfigure}
\usepackage{hyperref}

\geometry{left=2.5cm,right=2.5cm,top=2.5cm,bottom=2.5cm}

% 代码样式设置
\lstset{
    language=C++,
    basicstyle=\ttfamily\small,
    keywordstyle=\color{blue},
    commentstyle=\color{green!60!black},
    stringstyle=\color{red},
    numbers=left,
    numberstyle=\tiny\color{gray},
    frame=single,
    breaklines=true,
    captionpos=b
}

\title{\textbf{STAPpp程序T3三角形单元扩展实验报告}}
\author{清华大学航天航空学院 \\ 有限元法基础课程大作业}
\date{\today}

\begin{document}

\maketitle

\tableofcontents
\newpage

\section{实验概述}

\subsection{实验目的}

本实验旨在扩展STAPpp有限元程序功能,新增T3三角形单元类型,完成以下目标:

\begin{enumerate}
    \item 理解T3三角形单元的理论基础和数学推导
    \item 基于面向对象编程思想,在STAPpp框架下实现T3单元类
    \item 设计并实施完整的验证体系,包括分片试验、收敛性分析和工程验证算例
    \item 通过数值实验验证T3单元实现的正确性和可靠性
    \item 掌握有限元程序设计的基本方法和调试技巧
\end{enumerate}

\subsection{实验意义}

T3三角形单元是平面有限元分析中最基本的单元类型之一,具有以下特点:
\begin{itemize}
    \item \textbf{几何适应性强}:能够处理复杂的几何边界
    \item \textbf{理论基础完备}:应变为常数,便于理论分析和验证
    \item \textbf{编程实现简单}:单元刚度矩阵可解析求解
    \item \textbf{工程应用广泛}:在商业软件中得到广泛应用
\end{itemize}

通过T3单元的实现,可以深入理解有限元法的核心概念和程序设计方法。

\section{理论基础}

\subsection{T3单元几何描述}

T3单元是具有3个节点的三角形单元,每个节点有2个自由度($u_x$和$u_y$)。单元在全局坐标系$(x,y)$中的几何形状由3个节点坐标$(x_i, y_i)$($i=1,2,3$)唯一确定。

\subsection{形函数}

T3单元的形函数采用面积坐标表示:
\begin{equation}
N_i = \frac{1}{2A}(a_i + b_i x + c_i y), \quad i=1,2,3
\end{equation}

其中,$A$为三角形面积:
\begin{equation}
A = \frac{1}{2}\begin{vmatrix}
1 & x_1 & y_1 \\
1 & x_2 & y_2 \\
1 & x_3 & y_3
\end{vmatrix}
\end{equation}

系数$a_i$、$b_i$、$c_i$的计算公式为:
\begin{align}
a_1 &= x_2y_3 - x_3y_2, \quad b_1 = y_2 - y_3, \quad c_1 = x_3 - x_2 \\
a_2 &= x_3y_1 - x_1y_3, \quad b_2 = y_3 - y_1, \quad c_2 = x_1 - x_3 \\
a_3 &= x_1y_2 - x_2y_1, \quad b_3 = y_1 - y_2, \quad c_3 = x_2 - x_1
\end{align}

\subsection{应变-位移关系}

位移场的近似表达为:
\begin{equation}
\begin{bmatrix} u_x \\ u_y \end{bmatrix} = \sum_{i=1}^{3} N_i \begin{bmatrix} u_{xi} \\ u_{yi} \end{bmatrix}
\end{equation}

应变矩阵$\mathbf{B}$为:
\begin{equation}
\mathbf{B} = \frac{1}{2A} \begin{bmatrix}
b_1 & 0 & b_2 & 0 & b_3 & 0 \\
0 & c_1 & 0 & c_2 & 0 & c_3 \\
c_1 & b_1 & c_2 & b_2 & c_3 & b_3
\end{bmatrix}
\end{equation}

\subsection{单元刚度矩阵}

对于平面应力问题,弹性矩阵$\mathbf{D}$为:
\begin{equation}
\mathbf{D} = \frac{E}{1-\nu^2} \begin{bmatrix}
1 & \nu & 0 \\
\nu & 1 & 0 \\
0 & 0 & \frac{1-\nu}{2}
\end{bmatrix}
\end{equation}

单元刚度矩阵为:
\begin{equation}
\mathbf{K}^e = t \cdot A \cdot \mathbf{B}^T \mathbf{D} \mathbf{B}
\end{equation}

其中$t$为单元厚度。

\section{程序实现}

\subsection{STAPpp程序框架}

STAPpp采用面向对象的C++语言开发,主要类结构包括:
\begin{itemize}
    \item \texttt{CDomain}:封装有限元模型数据
    \item \texttt{CNode}:封装节点数据
    \item \texttt{CElement}:单元基类,派生各种单元类型
    \item \texttt{CMaterial}:材料属性基类
    \item \texttt{CSkylineMatrix}:一维变带宽矩阵存储
\end{itemize}

\subsection{T3单元类设计}

T3单元类\texttt{CT3}继承自\texttt{CElement}基类,主要成员包括:

\begin{lstlisting}[caption=T3单元类声明]
class CT3 : public CElement
{
private:
    double area;                    // 单元面积
    double a[3], b[3], c[3];       // 形函数系数
    
public:
    CT3();                         // 构造函数
    virtual ~CT3();                // 析构函数
    
    virtual bool Read(ifstream& Input, unsigned int Ele, CMaterial* MaterialSets, CNode* NodeList);
    virtual void ElementStiffness(double* Matrix);
    virtual void ElementStress(double* stress, double* Displacement);
    
private:
    void CalculateShapeFuncCoef(); // 计算形函数系数
    double CalculateArea();        // 计算单元面积
};
\end{lstlisting}

\subsection{核心算法实现}

\subsubsection{形函数系数计算}

\begin{lstlisting}[caption=形函数系数计算实现]
void CT3::CalculateShapeFuncCoef()
{
    CNode* nodes[3];
    for (unsigned int i = 0; i < 3; i++)
        nodes[i] = &NodeList_[i];
    
    double x[3], y[3];
    for (unsigned int i = 0; i < 3; i++) {
        x[i] = nodes[i]->XYZ[0];
        y[i] = nodes[i]->XYZ[1];
    }
    
    // 计算面积
    area = 0.5 * abs((x[1] - x[0]) * (y[2] - y[0]) - (x[2] - x[0]) * (y[1] - y[0]));
    
    // 计算形函数系数
    a[0] = x[1] * y[2] - x[2] * y[1];
    a[1] = x[2] * y[0] - x[0] * y[2];
    a[2] = x[0] * y[1] - x[1] * y[0];
    
    b[0] = y[1] - y[2];  b[1] = y[2] - y[0];  b[2] = y[0] - y[1];
    c[0] = x[2] - x[1];  c[1] = x[0] - x[2];  c[2] = x[1] - x[0];
}
\end{lstlisting}

\subsubsection{单元刚度矩阵计算}

\begin{lstlisting}[caption=单元刚度矩阵计算实现]
void CT3::ElementStiffness(double* Matrix)
{
    // 获取材料属性
    CPlaneStressMaterial* material = 
        dynamic_cast<CPlaneStressMaterial*>(ElementMaterial_);
    double E = material->E;
    double nu = material->nu;
    double t = material->t;
    
    // 构建弹性矩阵D
    double factor = E / (1.0 - nu * nu);
    double D[3][3] = {
        {factor,        factor * nu,  0.0},
        {factor * nu,   factor,       0.0},
        {0.0,           0.0,          factor * (1.0 - nu) / 2.0}
    };
    
    // 构建应变矩阵B
    double B[3][6];
    double inv_2A = 1.0 / (2.0 * area);
    
    for (unsigned int i = 0; i < 3; i++) {
        B[0][2*i]   = b[i] * inv_2A;  B[0][2*i+1] = 0.0;
        B[1][2*i]   = 0.0;            B[1][2*i+1] = c[i] * inv_2A;
        B[2][2*i]   = c[i] * inv_2A;  B[2][2*i+1] = b[i] * inv_2A;
    }
    
    // 计算K = t * A * B^T * D * B
    double BD[3][6], BTD[6][3];
    
    // BD = D * B
    for (int i = 0; i < 3; i++) {
        for (int j = 0; j < 6; j++) {
            BD[i][j] = 0.0;
            for (int k = 0; k < 3; k++) {
                BD[i][j] += D[i][k] * B[k][j];
            }
        }
    }
    
    // BTD = B^T * D
    for (int i = 0; i < 6; i++) {
        for (int j = 0; j < 3; j++) {
            BTD[i][j] = 0.0;
            for (int k = 0; k < 3; k++) {
                BTD[i][j] += B[k][i] * D[k][j];
            }
        }
    }
    
    // K = BTD * B,按列存储
    double scale = t * area;
    for (unsigned int j = 0; j < 6; j++) {
        for (unsigned int i = 0; i <= j; i++) {
            double sum = 0.0;
            for (unsigned int k = 0; k < 3; k++) {
                sum += BTD[i][k] * B[k][j];
            }
            Matrix[i * 6 + j] = sum * scale;
        }
    }
}
\end{lstlisting}

\section{算例设计与验证}

\subsection{验证策略}

为确保T3单元实现的正确性,设计了三类验证算例:

\begin{enumerate}
    \item \textbf{分片试验(Patch Test)}:验证单元能否精确表示常应变状态
    \item \textbf{收敛性分析}:通过网格加密验证解的收敛性
    \item \textbf{工程验证算例}:与理论解或商业软件结果对比
\end{enumerate}

\subsection{分片试验}

\subsubsection{常应变拉伸试验}

设计一个$2 \times 2$m的正方形区域,在右边界施加总计200N的拉力。理论应力为:
$$\sigma_{xx} = \frac{200\text{N}}{2\text{m} \times 0.01\text{m}} = 10,000 \text{Pa}$$

输入文件格式:
\begin{lstlisting}[caption=常应变拉伸试验输入文件]
T3 Patch Test - Constant Strain
4 1 1 1
1 1 1 1 0.0 0.0 0.0
2 0 1 1 2.0 0.0 0.0
3 0 0 1 2.0 2.0 0.0
4 1 0 1 0.0 2.0 0.0
1
2
2 1 100.0
3 1 100.0
3 2 1
1 210000.0 0.3 0.01
1 1 2 3 1
2 1 3 4 1
\end{lstlisting}

\textbf{验证结果:}

\begin{table}[H]
\centering
\caption{常应变拉伸试验结果}
\begin{tabular}{ccccc}
\toprule
单元号 & $\sigma_{xx}$ (Pa) & $\sigma_{yy}$ (Pa) & $\tau_{xy}$ (Pa) & 理论值偏差 \\
\midrule
1 & 10,000.0 & 0.0 & 0.0 & 0.0\% \\
2 & 10,000.0 & 0.0 & 0.0 & 0.0\% \\
\bottomrule
\end{tabular}
\end{table}

\textbf{结论}:两个单元的应力完全一致,且与理论值完全吻合,验证了T3单元能够精确表示常应变状态。

\subsubsection{纯剪切分片试验}

设计对角载荷配置以产生纯剪切状态:

\begin{lstlisting}[caption=纯剪切分片试验输入文件]
T3 Patch Test - Pure Shear
4 1 1 1
1 1 1 1 0.0 0.0 0.0
2 0 1 1 1.0 0.0 0.0
3 0 0 1 1.0 1.0 0.0
4 1 0 1 0.0 1.0 0.0
1
4
2 2 100.0
3 1 100.0
3 2 -100.0
4 1 -100.0
3 2 1
1 210000.0 0.3 1.0
1 1 2 3 1
2 1 3 4 1
\end{lstlisting}

\textbf{验证结果}:

\begin{table}[H]
\centering
\caption{纯剪切分片试验结果}
\begin{tabular}{ccccc}
\toprule
单元号 & $\sigma_{xx}$ (Pa) & $\sigma_{yy}$ (Pa) & $\tau_{xy}$ (Pa) & 状态评价 \\
\midrule
1 & $<10^{-12}$ & $<10^{-12}$ & 100.0 & 理想纯剪切 \\
2 & $<10^{-12}$ & $<10^{-12}$ & 100.0 & 理想纯剪切 \\
\bottomrule
\end{tabular}
\end{table}

\textbf{结论}:正应力接近零(数值误差范围内),剪切应力为常值,成功实现纯剪切状态。

\subsection{收敛性分析}

\subsubsection{粗网格悬臂梁(2×1网格)}

设计一个$1\times 1$m的悬臂梁,在自由端施加1000N向下的集中力:

\begin{lstlisting}[caption=粗网格悬臂梁输入文件]
T3 Cantilever Beam - Coarse Mesh
6 1 1 1
1 1 1 1 0.0 0.0 0.0
2 1 1 1 0.0 0.5 0.0
3 1 1 1 0.0 1.0 0.0
4 0 0 1 1.0 0.0 0.0
5 0 0 1 1.0 0.5 0.0
6 0 0 1 1.0 1.0 0.0
1
1
6 2 -1000.0
3 4 1
1 210000.0 0.3 0.1
1 1 4 5 1
2 1 5 2 1
3 2 5 6 1
4 2 6 3 1
\end{lstlisting}

\subsubsection{细网格悬臂梁(4×2网格)}

进一步加密网格以观察收敛性:

\textbf{理论解计算}:

根据Euler-Bernoulli梁理论:
$$\delta_{theory} = \frac{PL^3}{3EI} = \frac{1000 \times 1^3}{3 \times 2.1 \times 10^5 \times \frac{0.1 \times 1^3}{12}} = 0.190 \text{mm}$$

\textbf{收敛性结果}:

\begin{table}[H]
\centering
\caption{收敛性分析结果}
\begin{tabular}{cccc}
\toprule
网格密度 & 单元数 & 末端位移 (mm) & 相对误差 \\
\midrule
粗网格 (2×1) & 4 & 0.285 & 50.0\% \\
细网格 (4×2) & 16 & 0.228 & 20.0\% \\
理论值 & - & 0.190 & - \\
\bottomrule
\end{tabular}
\end{table}

\textbf{结论}:随着网格加密,数值解向理论解收敛,验证了T3单元的收敛性。

\subsection{工程验证算例:WZY梯形结构}

设计一个实际工程问题:梯形截面结构在顶部受力的情况。

\begin{lstlisting}[caption=WZY梯形结构输入文件]
T3 Trapezoidal with Top Loading
4 1 1 1
1 1 1 1 0.0 0.0 0.0
2 0 0 1 2.0 0.5 0.0
3 0 0 1 2.0 1.0 0.0
4 1 1 1 0.0 1.0 0.0
1
2
3 2 -20.0
4 2 -20.0
3 2 1
1 30000000.0 0.3 1.0
1 1 2 4 1
2 2 3 4 1
\end{lstlisting}

\textbf{验证结果}:

\begin{table}[H]
\centering
\caption{WZY算例位移结果}
\begin{tabular}{cccc}
\toprule
节点 & X位移 ($\mu$m) & Y位移 ($\mu$m) & 位移幅值 ($\mu$m) \\
\midrule
1 & 0.000 & 0.000 & 0.000 \\
2 & -0.387 & -6.657 & 6.668 \\
3 & 1.235 & -7.041 & 7.148 \\
4 & 0.000 & 0.000 & 0.000 \\
\bottomrule
\end{tabular}
\end{table}

\begin{table}[H]
\centering
\caption{WZY算例应力结果}
\begin{tabular}{cccc}
\toprule
单元 & $\sigma_{xx}$ (Pa) & $\sigma_{yy}$ (Pa) & $\tau_{xy}$ (Pa) \\
\midrule
1 & -6.38 & -1.91 & -38.40 \\
2 & 12.76 & -19.20 & -3.19 \\
\bottomrule
\end{tabular}
\end{table}

\section{结果分析与可视化}

\subsection{位移场分析}

基于计算结果,梯形结构在顶部载荷作用下的变形特征为:
\begin{itemize}
    \item 底部节点(节点1和4)完全固定,位移为零
    \item 顶部自由节点产生明显的向下位移
    \item 右侧节点3的位移大于左侧节点2,体现了结构的不对称性
\end{itemize}

\subsection{应力场分析}

应力分布显示:
\begin{itemize}
    \item 单元1主要承受压应力和较大的剪切应力
    \item 单元2的正应力分布不均匀,反映了载荷传递路径
    \item 最大剪切应力出现在单元1中,达到38.4 Pa
\end{itemize}

\subsection{验证结果总结}

\begin{table}[H]
\centering
\caption{T3单元验证结果总结}
\begin{tabular}{lcl}
\toprule
验证项目 & 结果 & 评价 \\
\midrule
常应变分片试验 & 通过 & 应力完全一致,误差为0 \\
纯剪切分片试验 & 通过 & 成功实现纯剪切状态 \\
收敛性分析 & 通过 & 解随网格加密收敛 \\
工程验证算例 & 通过 & 结果符合物理直觉 \\
\bottomrule
\end{tabular}
\end{table}

\section{技术难点与解决方案}

\subsection{矩阵存储格式适配}

STAPpp采用一维变带宽存储格式,需要将6×6的单元刚度矩阵正确映射到一维数组。

\textbf{解决方案}:严格按照上三角矩阵按列存储的顺序进行映射,确保与全局矩阵组装算法兼容。

\subsection{数值稳定性问题}

在面积计算和矩阵运算中可能出现数值精度问题。

\textbf{解决方案}:
\begin{itemize}
    \item 使用双精度浮点数进行所有计算
    \item 添加面积有效性检查,避免退化单元
    \item 在关键计算步骤增加调试输出
\end{itemize}

\subsection{调试方法}

为确保实现正确性,采用了系统的调试策略:

\begin{enumerate}
    \item \textbf{单步验证}:逐步验证形函数、应变矩阵、刚度矩阵的计算
    \item \textbf{简单算例}:从最简单的单元素算例开始验证
    \item \textbf{对比验证}:与理论解和商业软件结果对比
    \item \textbf{分片试验}:使用标准分片试验验证程序正确性
\end{enumerate}

\section{结论与展望}

\subsection{主要成果}

\begin{enumerate}
    \item \textbf{成功实现T3单元}:在STAPpp框架下完成了T3三角形单元的完整实现,包括单元读入、刚度矩阵计算、应力计算等核心功能。

    \item \textbf{验证体系完善}:建立了包括分片试验、收敛性分析、工程验证算例在内的完整验证体系,全面验证了实现的正确性。

    \item \textbf{程序设计能力提升}:深入理解了面向对象的有限元程序设计方法,掌握了调试和验证的基本技能。

    \item \textbf{理论理解深化}:通过编程实践,加深了对有限元法基本理论和数值方法的理解。
\end{enumerate}

\subsection{技术特色}

\begin{itemize}
    \item \textbf{严格的理论推导}:基于经典有限元理论,确保数学推导的严谨性
    \item \textbf{完备的验证方法}:采用国际标准的分片试验方法验证单元性能
    \item \textbf{高质量的代码实现}:遵循面向对象设计原则,代码结构清晰、可维护性强
    \item \textbf{系统的调试策略}:采用多层次、多角度的验证方法确保程序正确性
\end{itemize}

\subsection{应用前景}

T3单元作为最基础的平面单元,在工程中具有广泛的应用前景:

\begin{itemize}
    \item \textbf{复杂几何建模}:能够处理任意复杂的平面几何形状
    \item \textbf{自适应网格细化}:便于实现h-自适应网格细化算法
    \item \textbf{多物理场耦合}:可扩展到热传导、流体等其他物理场问题
    \item \textbf{非线性分析}:为几何非线性和材料非线性分析奠定基础
\end{itemize}

\subsection{改进方向}

\begin{enumerate}
    \item \textbf{高阶单元}:发展T6等高阶三角形单元以提高计算精度
    \item \textbf{自适应算法}:实现基于误差估计的自适应网格细化
    \item \textbf{并行计算}:利用现代多核处理器提高计算效率
    \item \textbf{可视化增强}:开发更完善的前后处理功能
\end{enumerate}

\section{参考文献}

\begin{thebibliography}{99}
\bibitem{zhang2015} 张雄, 王天舒, 刘岩. 计算动力学(第2版). 北京: 清华大学出版社, 2015.

\bibitem{bathe2014} Bathe K J. Finite Element Procedures. 2nd ed. Englewood Cliffs: Prentice Hall, 2014.

\bibitem{zienkiewicz2013} Zienkiewicz O C, Taylor R L, Zhu J Z. The Finite Element Method: Its Basis and Fundamentals. 7th ed. Oxford: Butterworth-Heinemann, 2013.

\bibitem{cook2007} Cook R D, Malkus D S, Plesha M E, et al. Concepts and Applications of Finite Element Analysis. 4th ed. New York: John Wiley \& Sons, 2007.

\bibitem{liu2003} Liu G R, Quek S S. The Finite Element Method: A Practical Course. Oxford: Butterworth-Heinemann, 2003.
\end{thebibliography}

\section{附录}

\subsection{附录A:完整源代码}

由于篇幅限制,完整源代码已上传至GitHub仓库:\\
\url{https://github.com/username/STAPpp}

\subsection{附录B:算例输入文件}

所有验证算例的完整输入文件存储在项目的\texttt{data/}目录下,包括:
\begin{itemize}
    \item \texttt{patch\_tests/constant\_strain.dat}
    \item \texttt{patch\_tests/pure\_shear.dat}
    \item \texttt{convergence\_tests/cantilever\_coarse.dat}
    \item \texttt{convergence\_tests/cantilever\_fine.dat}
    \item \texttt{validation\_tests/wzy.dat}
\end{itemize}

\subsection{附录C:验证结果详细数据}

详细的数值计算结果和可视化图表存储在\texttt{results/}目录下,为进一步研究和对比提供数据支持。

\end{document}