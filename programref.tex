\documentclass{article}
\usepackage{amsmath}
\usepackage{amssymb}
\usepackage{graphicx}
\usepackage{booktabs}
\usepackage{multirow}

\title{直接刚度法 - 有限元程序设计}
\author{清华大学}
\date{2025/2/27}

\begin{document}

\maketitle

\section{第2章 直接刚度法}
\subsection{2.5 有限元程序设计}

\subsubsection{有限元程序设计概述}
- 有限元法具有很强的通用性,可分析几乎任何具有复杂边界和加载条件的连续介质问题。
- 有限元程序的效率取决于{insert\_element\_0\_}所使用的单元/算法和程序设计方法。
- 现有有限元商用软件大部{insert\_element\_1\_}分采用面向过程的FORTRAN语言,这是历史原因导致的。
- C++语言是面向对象的{insert\_element\_2\_}语言,支持数据封装和隐藏、继承和重用以及多态性。
- C++程序具有良好的可{insert\_element\_3\_}维护性、易修改性和复用性,许多大型科学计算软件都采用或改用C++语言。
- 开源计算流体力学软件O{insert\_element\_4\_}penFOAM是用C++开发的。
- 分子动力学模拟开源程序{insert\_element\_5\_}包LAMMPS先后采用Fortran 77和Fortran 90开发,2004年后改用C++开发。

\subsubsecti{insert\_element\_6\_}on{STAP系列程序}
- STAP程序是K.J. Bathe用FORTRAN IV编写的,其程序结构与SAP和ADINA的程序结构类似。
- STAP90是用FOR{insert\_element\_7\_}TRAN 90语言对STAP程序改写而成的,作为《计算动力学》教材的教学程序,可在GitHub上获取。
- STAPpp是用C++{insert\_element\_8\_}语言编写的基于面向对象思想设计的有限元法程序,可在GitHub上获取。
- STAPpy是用Pyt{insert\_element\_9\_}hon语言编写的基于面向对象思想设计的有限元法程序,可在GitHub上获取。
- STAPpp只包含有限{insert\_element\_10\_}元法的中间三步:单元分析、总刚组装和方程求解。
- 前处理可使用Gmsh({insert\_element\_11\_}开源)、HyperMesh、GiD等软件,后处理可使用Tecplot、ParaView(开源)等软件。
- STAP90、STAP{insert\_element\_12\_}pp和STAPpy的输入输出文件格式相同,大作业1可以选择其中之一。

\subsubsecti{insert\_element\_13\_}on{STAPpp程序类}
- CDomain类:对有限元模型数据进行封装。
- CNode类:封装节点{insert\_element\_14\_}数据。
- CElement抽象类{insert\_element\_15\_}:封装单元数据,派生各类单元。
- CElementGro{insert\_element\_16\_}up类:封装单元组数据,所有单元被分为若干个单元组,每个单元组只允许具有一种类型的单元。
- CMaterial抽象{insert\_element\_17\_}类:封装截面/材料属性数据,派生各类截面/材料。
- CLoadCaseDa{insert\_element\_18\_}ta类:封装载荷数据。
- CSkylineMat{insert\_element\_19\_}rix类:封装按一维变带宽方式存储的总体刚度矩阵。

\subsubsecti{insert\_element\_20\_}on{有限元模型数据}
\paragraph{节点数据}
- 一个有限元节点应具有节点号、节点坐标和节点边界条件等数据。
- 每个节点各自由度边界条{insert\_element\_21\_}件类型由bcode[NDF]表示,其中bcode[i]=0表示该节点的第i个方向自由,bcode[i]=1表示该节点的第i个方向固定。
- NDF为每个节点的最大{insert\_element\_22\_}自由度数,例如平面应力/应变单元NDF=2,板单元NDF=3,壳单元NDF=5/6。

\paragraph{单{insert\_element\_23\_}元数据}
- 描述一个单元所需的数据取决于单元的具体类型,包括单元的节点编号、截面/材料属性、施加在该单元上的面力或体力等🔶2-61🔶🔶2-62🔶🔶2-63🔶🔶2-64🔶🔶2-65🔶。
- 多个单元可以具有相同的单元截面/材料属性和单元载荷,单元数据通常以表格形式存储,例如:

\begin{center}
\begin{tabular}{|c|c|c|c|}
\hline
单元号 & 左节点号 & 右节点号 & 截面/材料属性组号 \\
\hline
1 & 1 & 7 & 1 \\
\hline
2 & 1 & 2 & 1 \\
\hline
3 & 2 & 7 & 1 \\
\hline
$\vdots$ & $\vdots$ & $\vdots$ & $\vdots$ \\
\hline
22 & 6 & 12 & 1 \\
\hline
\end{tabular}
\end{center}

\paragraph{载荷数据}
- 在有限元法中,载荷既可以作用在单元上,也可以直接作用在节点上,但最终都会转化为节点载荷求解。
- STAPpp目前只允许{insert\_element\_24\_}输入节点载荷,作用在单元上的载荷需要由用户自己转化为节点载荷,用户也可扩充代码以允许输入单元载荷。
- 载荷数据以表格形式输入{insert\_element\_25\_},例如:

\begin{center}
\begin{tabular}{|c|c|c|}
\hline
受载荷作用的节点号 & 载荷作用方向 & 载荷值 \\
\hline
3 & 2 & -7000 \\
\hline
\end{tabular}
\end{center}

\subsubsection{输入数据文件格式}
STAPpp的输入数据文件包括以下五部分,需严格按照顺序填写:

\paragraph{标题行}
- 列1-80为标题HED(80),用于对所求解问题进行简单描述。

\paragraph{控{insert\_element\_26\_}制行}
- 列1-5为NUMNP,即结点总数,若为0则程序终止运行。
- 列6-10为NUMFG{insert\_element\_27\_},即单元组总数,每个单元组只能包含相同类型的单元。
- 列11-15为NLCA{insert\_element\_28\_}SE,即载荷工况数。
- 列16-20为MODE{insert\_element\_29\_}X,即求解模式,等于0时只做数据检查,等于1时进行求解。

\paragraph{结{insert\_element\_30\_}点数据}
- 列1-5为结点号N。
- 列6-10为ID(1,{insert\_element\_31\_}N),即x-平动方向边界条件代码。
- 列11-15为ID(2{insert\_element\_32\_},N),即y-平动方向边界条件代码。
- 列16-20为ID(3{insert\_element\_33\_},N),即z-平动方向边界条件代码。
- 列21-30为X(N){insert\_element\_34\_},即x-坐标。
- 列31-40为Y(N){insert\_element\_35\_},即y-坐标。
- 列41-50为Z(N){insert\_element\_36\_},即z-坐标。

\paragraph{载{insert\_element\_37\_}荷数据}
- 先输入NLCASE组载荷数据控制行,列1-5为载荷工况号LL,列6-10为NLOAD,即本工况中集中载荷的个数。
- 再输入各工况载荷数据,{insert\_element\_38\_}列1-5为NOD,即集中载荷作用的结点号;列6-10为IDIRN,即载荷作用方向;列11-20为FLOAD,即载荷值。

\paragraph{轴{insert\_element\_39\_}力杆单元数据}
- 单元组控制数据:列1-5为NPAR(1),即单元类型;列6-10为NPAR(2),即本单元组中的单元总数;列11-15为NPAR(3),即不同材料/截面性质组数🔶2-117🔶。
- 材料/截面性质数据:列{insert\_element\_40\_}1-5为材料/截面性质组号N;列6-15为E(N),即杨氏模量;列16-25为AREA(N),即截面面积🔶2-120🔶。
- 单元数据:列1-5为单{insert\_element\_41\_}元号M;列6-10为II,即单元左端结点号;列11-15为JJ,即单元右端结点号;列16-20为MTYP,即本单元的材料/截面性质组号🔶2-123🔶。

\subsubsecti{insert\_element\_42\_}on{结构总体矩阵的存储与组装}
\paragraph{存储格式}
- 结构总体刚度矩阵是高度稀疏的,常用的存储格式包括一维变带宽存储格式、行压缩格式(CSR)、列压缩格式(CSC)、坐标存储格式(COO)等🔶2-133🔶🔶2-136🔶🔶2-137🔶🔶2-138🔶。
- 一维变带宽存储格式用一个一维数组A按列依次存储结构总体刚度矩阵的上三角阵,可大大减少存储量和计算时间。

\paragraph{关{insert\_element\_43\_}键概念}
- 列高$H_i$:矩阵K第i列第1个非零元素的行号$m_i$,则列高$H_i = i - m_i$,表示第i列轮廓线内元素的个数(不含对角元)。
- 最大半带宽$m_k ={insert\_element\_44\_} \max(i - m_i)$。
- 对角元地址数组Diag{insert\_element\_45\_}:各对角元在数组A中的地址为前$(i-1)$列的列高的总和再加上i。
- $K_{ij}(j \{insert\_element\_46\_}geq i)$的地址为$M_j + (j - i)$,其中$M_i = \sum_{j=1}^{i-1} H_j + i$。

\paragraph{矩{insert\_element\_47\_}阵组装}
- 单元对号数组LocationMatrix存储单元各自由度所对应的总体自由度号,可通过bcode生成。
- 结构总体矩阵的组装公式{insert\_element\_48\_}为$K = \sum_{e} L^{eT} K^{e} L^{e}$。

\paragraph{节点重编号}
- 对节点进行重编号可以缩减列高,应使所有相邻结点对的编号差的最大绝对值尽可能小🔶2-183🔶。

\subsubsecti{insert\_element\_49\_}on{STAPpp文件结构}
STAPpp文件结构包括多个C++源文件和头文件,如Bar.cpp、Clock.cpp、Domain.cpp、Element.cpp、ElementGroup.cpp、LoadCaseData.cpp、Node.cpp、Material.cpp、Solve.cpp等,以及CMakeLists.txt用于自动生成工程文件🔶2-205🔶。

\subsubsecti{insert\_element\_50\_}on{练习与填空题}
\paragraph{练习}
一桁架结构由5个杆组成,节点1和2自由,其余节点固定,各节点的边界条件类型数组bcode和各单元的节点号数组IEN分别为:
\[
bcode = \left[\begin{array}{lllll}0 & 0 & 1 & 1 & 1 \\ 0 & 0 & 1 & 1 & 1\end{array}\right]
\]
\[
IEN = \left[\begin{array}{lllll}3 & 4 & 1 & 4 & 5 \\ 1 & 1 & 2 & 2 & 2\end{array}\right]
\]
求各节点自由度对应的总体自由度号、各单元对号数组、总刚度阵各列列高、对角元地址和数组A的大小。

\paragraph{填空题}
- 对于给定的bcode矩阵,第20列列高为[填空1]🔶2-210🔶。
- 已知bcode和IEN{insert\_element\_51\_},求M(:) = [填空1], [填空2], [填空3], [填空4], [填空5];H(:) = [填空5], [填空6], [填空7], [填空8];LM(:,4) = [填空8], [填空9];数组A非零元素的个数 = [填空1]🔶2-215🔶🔶2-216🔶🔶2-217🔶🔶2-218🔶🔶2-219🔶🔶2-220🔶🔶2-221🔶🔶2-222🔶。

\subsubsection{C++面向对象编程基础}
- C++在C语言的基础上增加了面向对象编程,类是C++的核心特性,用于指定对象的形式,包含数据表示法和处理数据的方法🔶2-231🔶。
- 类的定义示例:
\be{insert\_element\_52\_}gin{verbatim}
class Box {
public:
    double length;   // 长度
    double breadth;  // 宽度
    double height;   // 高度
    double get(void);
    void set(double len, double bre, double hei);
};

double Box::get(void) {
    return length * breadth * height;
}

void Box::set(double len, double bre, double hei) {
    length = len;
    breadth = bre;
    height = hei;
}
\end{verbatim}

\end{document}